\documentclass[aspectratio=43,xcolor=table]{beamer}
\usepackage[utf8]{inputenc}

% Special symbol
\usepackage{keystroke} % Keyboard
\usepackage{fontawesome} % Internet emoji
\usepackage{tikzsymbols} 
\usepackage{textcomp} % Arrows in text mode

% Font family
\usepackage{palatino}

% Tables
\usepackage[table,xcdraw]{xcolor}
\arrayrulecolor{White}

% Import theme
% Style file to create my own beamer theme

\usepackage[absolute,overlay]{textpos}
\usepackage{graphicx} % Allows including images
\usepackage{booktabs} % Allows the use of \toprule, \midrule and \bottomrule in tables
\usepackage{tikz,pgfplots}
\usepackage{pgfpages,lipsum}
\usepackage[pages=some]{background}
\usetikzlibrary{shapes, snakes, arrows}
\usepackage{xcolor}
% Set spacing
\setlength{\parskip}{10pt plus 1pt minus 1pt}

% Multiple audiences presentation mode
\usepackage{multiaudience}

% Declare all possible audience groups
\SetNewAudience{All}
\SetNewAudience{OnlyMe}

\DefCurrentAudience{All}

%%% Defining the preamble
\mode

%%% Defining inner theme
% Here we am using the rounded theme for the overall “feel” of the presentation
\useinnertheme{rounded}

% Instead of rounded circles, we will use triangles as the indicator for items.
\setbeamertemplate{itemize items}[triangle]

% The default option for enumerate environment removes the circles around the numbers provided by the “rounded” inner theme. Just simple numbers remain.
%\setbeamertemplate{enumerate items}[default]

%%% Defining colors
% SU palette colors
\definecolor{Darkblue}{RGB}{0, 47, 95}
\definecolor{Eld}{RGB}{235, 113, 37}
\definecolor{Oliv}{RGB}{163, 168, 107}
\definecolor{Vatten}{RGB}{155, 178, 206}
\definecolor{Himmel}{RGB}{61, 216, 224}
\definecolor{White}{RGB}{255, 255, 255}

% Self define colors
\definecolor{AlertRed}{RGB}{220, 85, 95}
\definecolor{Teal}{RGB}{87, 176, 181}


% Colors for constituents
\definecolor{Text1}{RGB}{0, 0, 0}
\definecolor{Descitem}{RGB}{55, 100, 137}

% Defining colors for blocks
\definecolor{StdTitle}{RGB}{255, 160, 0}
\definecolor{StdBody}{RGB}{255, 160, 0}

\definecolor{AlTitle}{RGB}{220, 85, 95}
\definecolor{AlBody}{RGB}{220, 85, 95}

\definecolor{ExTitle}{RGB}{87, 176, 181}
\definecolor{ExBody}{RGB}{87, 176, 181}

% Assign colours to different constituents of the presentation
\setbeamercolor{frametitle}{fg = Darkblue}
\setbeamercolor{title}{fg = Darkblue}
\setbeamercolor{normal text}{fg = Text1}

% In case you choose to display the Table of Contents, or the Outline slide.
\setbeamercolor{section in toc}{fg = Text1}
\setbeamercolor{section in toc shaded}{fg = Text1}

% The colour of all the items, subitems and and subsubitems are set to black.
\setbeamercolor{item}{fg = Darkblue}
\setbeamercolor{subitem}{fg = Darkblue}
\setbeamercolor{subsubitem}{fg = Darkblue}

\setbeamertemplate{itemize item}[\textbullet]
\setbeamertemplate{itemize subitem}[\textasteriskcentered]
\setbeamertemplate{itemize subsubitem}[circle]
% This sets the color for each item heading of the description environment.
\setbeamercolor{description item}{fg = Descitem}

% We also need to fix the colours for captions for figures and tables.
\setbeamercolor{caption}{fg = Ftitle}
\setbeamercolor{caption name}{fg = Ftitle}

% In addition, we can also change the background colour of the slides depending on pur requirement.
% \setbeamercolor{background canvas}{bg = blue!5}

% We will customize all these blocks based on our requirements

% Standard block

\setbeamercolor{block title}{fg = White, bg = StdTitle}
\setbeamercolor{block body}{bg = StdBody!15!White}
\AtBeginEnvironment{block}{%
    \setbeamercolor{itemize item}{fg=StdTitle}
    \setbeamercolor{itemize subitem}{fg=StdTitle}
    \setbeamercolor{itemize subsubitem}{fg=StdTitle}
    \setbeamercolor{enumerate item}{fg=StdTitle, bg=White}
    \setbeamercolor{enumerate subitem}{fg=StdTitle}
    \setbeamercolor{enumerate subsubitem}{fg=StdTitle}
    \setbeamercolor{item projected}{fg=white, bg=StdTitle}
}

% Alert block

\setbeamercolor{block title alerted}{fg = White, bg = AlTitle}
\setbeamercolor{block body alerted}{bg = AlBody!15!White}
\AtBeginEnvironment{alertblock}{%
    \setbeamercolor{itemize item}{fg=AlTitle}
    \setbeamercolor{itemize subitem}{fg=AlTitle}
    \setbeamercolor{itemize subsubitem}{fg=AlTitle}
    \setbeamercolor{enumerate item}{fg=AlTitle, bg=White}
    \setbeamercolor{enumerate subitem}{fg=AlTitle}
    \setbeamercolor{enumerate subsubitem}{fg=AlTitle}
    \setbeamercolor{item projected}{fg=White, bg=AlTitle}
}

% Example block

\setbeamercolor{block title example}{fg = White, bg = ExTitle}
\setbeamercolor{block body example}{bg = ExBody!15!White}

\AtBeginEnvironment{exampleblock}{%
    \setbeamercolor{itemize item}{fg=ExTitle}
    \setbeamercolor{itemize subitem}{fg=ExTitle}
    \setbeamercolor{itemize subsubitem}{fg=ExTitle}
    \setbeamercolor{enumerate item}{fg=ExTitle, bg=White}
    \setbeamercolor{enumerate subitem}{fg=ExTitle}
    \setbeamercolor{enumerate subsubitem}{fg=ExTitle}
    \setbeamercolor{item projected}{fg=white, bg=ExTitle}
}

%%% Shortcut for creating each slide
\newcommand{\mframe}[3][All]{
\begin{shownto}{#1}
    \begin{frame}
        \begin{textblock}{3}(14.5,0.5)
            \begin{tikzpicture}[x=1mm,y=1mm]
        	 \foreach \x in {0,1,...,\inserttotalframenumber-1} 
        	 					\draw [gray!30,very thick,cap=round] (90-\x/\inserttotalframenumber*360:2.5) -- (90-\x/\inserttotalframenumber*360:3.5);
           \foreach \x in {0,...,\insertframenumber-1} 
           					\draw[Darkblue,very thick,cap=round] (90-\x/\inserttotalframenumber*360:2.5) -- (90-\x/\inserttotalframenumber*360:3.5);
           \node at (0,0) [] {\tiny{\textbf{\insertframenumber}}};
        	\end{tikzpicture}
        \end{textblock}
        
        % In case you need to move the logo to up right
        %\begin{textblock}{3}(13,0.5)		
        %\centering
        %\includegraphics[0.5\textwidth]{SUlogo.jpg}
        %\end{textblock}
        \frametitle{#2}
        #3
    \end{frame}
\end{shownto}
}

%%% Set font theme
% Here we are using default fonts
\usefonttheme{serif}

% Font for the presentation title
\setbeamerfont{title}{size = \Large}
\setbeamerfont{subtitle}{size = \huge}

% Font of the frame titles
\setbeamerfont{frametitle}{size = \Large}

% Font for other title page information
\setbeamerfont{author}{size = \large}
\setbeamerfont{date}{size = \normalsize,}
\setbeamerfont{institute}{size = \normalsize,}

%%% Defining outer theme
% remove the navigation symbols
\setbeamertemplate{navigation symbols}{}

% Organize the presentation title and the frame titles

% Effects setting
\setbeamercovered{transparent}


% Information in the title page
\title{Empirical Finance, Fall 2019}
\subtitle {Computer Exercise 1}
\author{Chengcheng Qu}
\date{\today}
\logo{\includegraphics[0.75\textwidth]{SUlogo.jpg}}
\institute{Stockholm Business School}
 
 % New command for boxed text
 \newcommand{\key}[1]{\fbox{\textbf{#1}}}

\begin{document}
 
\begin{frame}[noframenumbering]
\begin{tikzpicture}[remember picture, overlay]
      \node[anchor = north west, opacity = 0.15] at (-1.75,1) {
      \pgfimage[width=\textwidth]{Olive.jpg}};
\end{tikzpicture}
\titlepage 
\end{frame}

\mframe{Excel: Advantages \& Limits}{
Excel is straightforward and multifunctional. The user can view, edit, visualize, and analyze data (numeric, text, date, logical value) directly and spontaneously.
\pause

But it can be painful when \vspace{-10pt}
\begin{itemize}
	\item[\Walley] a large data crashes Excel \pause
	\item[\Walley] an accidental change happens but Excel can't undo it \pause
	\item[\Walley] an analysis contains many steps
\end{itemize}
\pause

Some companies are switching from Excel to SQL, business intelligence tools, or other programming languages
}

\mframe{Outline}{

\begin{itemize}
\setlength\itemsep{1em}
	\item Excel function: always starts with \key{=} \pause
	
	\item Absolute vs. relative reference: 
	
	switch among \textcolor{Darkblue}{A1},  \textcolor{Darkblue}{\$A\$1},  \textcolor{Darkblue}{A\$1},  \textcolor{Darkblue}{\$A1} with \key{F4} 
	\pause
	
    \item Name manager \key{Ctrl} + \key{F3}
    \pause
    
    \item Matrix calculation in Excel:
	
	use \key{Ctrl} + \key{Shift} + \key{Enter}
    
\end{itemize}
}

\mframe{Introduction to Functions}{

\textcolor{Eld}{\faLightbulbO} Let Excel know it is a function with \key{=} \vspace{-10pt}
\pause

\textcolor{Eld}{\faLightbulbO} Excel function name is not case-sensitive
\pause

PI(): Returns the value of $\pi$

ABS(x): Returns the absolute value of x \vspace{-10pt}

SQRT(x): Returns the square root of x
\pause

\begin{examples}
\begin{itemize}
    \item[\textbullet] =PI()  \Return or =pi()  \Return obtains 3.141593
    \item[\textbullet] Given \textcolor{ExTitle}{A1}=4, =SQRT(\textcolor{ExTitle}{A1}) \Return obtains 2
    \item[\textbullet] Given \textcolor{ExTitle}{A1}=-4, =SQRT(\textcolor{ExTitle}{A1}) \Return obtains 
\textcolor{AlTitle}{\#NUM!}
\end{itemize}
\end{examples}
}

\mframe{Introduction to Functions (cont.)}{
LOG(x, base): Returns the logarithm of x to a given base \vspace{-10pt}
\pause

LN(x): Returns the natural logarithm of x \vspace{-10pt}
\pause

POWER(x, power) and EXP(x) are the exponential functions 

\pause
\textcolor{Eld}{\faLightbulbO} Function separator could be ``;" instead of ``,"
\pause

\begin{block}{Shortcut}
\begin{itemize}
    \item[\textbullet] Fill in along column(row) with fill handle \textbf{+}
    \item[\textbullet] Use absolute reference \textcolor{Eld}{\$} so the value doesn't "move"
\end{itemize}
\end{block}

\pause

\textcolor{Eld}{\faLightbulbO} The argument x can also be numbers, arrays, or names 

SUM(x): Sum up of values in x  \vspace{-10pt}
}

\mframe{Data Analysis and Statistics}{
Functions connect analysis results with data dynamically

AVERAGE(x): Returns the average of values in x \vspace{-10pt}

STDEV.P(x) or STDEV.S(x): Returns the population (or sample) standard deviation of values in x \vspace{-10pt}

MIN(x): Returns the minimum value of values in x  \vspace{-10pt}

MAX(x): Returns the maximum value of values in x  

NORM.DIST(x, $\mu$, $\sigma$, Cumulative): \vspace{-10pt}

If Cumulative is \textcolor{ExTitle}{TRUE} returns cumulative distribution function; if \textcolor{ExTitle}{FALSE} returns probability mass function \vspace{-10pt}

NORM.INV(Prob, $\mu$, $\sigma$): \vspace{-10pt}

Returns the inverse of the normal cumulative distribution

INTERCEPT(y, x): returns the intercept of a regression \vspace{-10pt}

SLOPE (y, x): returns the slope of a regression

}

\mframe{Data Analysis and Statistics (cont.)}{

An alternative way to do statistic analysis is Data Analysis Add-in

File \textrightarrow Options \textrightarrow Add-Ins \textrightarrow Analysis ToolPak \textrightarrow \key{Go}
\textrightarrow Data Analysis Add-in on Toolbar

We will come back to this part in following seminars
}

\mframe{Names}{
An index like \textcolor{Darkblue}{A1:Z100} can be annoying to select and track...

Instead we can use a \textcolor{Darkblue}{NAME} for \textcolor{Darkblue}{A1:Z100}. 
To define, view, and edit names: \vspace{-10pt}

Formulas \textrightarrow Name Manager

\begin{block}{Shortcut}
\begin{itemize}
    \item[\textbullet] Select a region then right-click to define a name
    \item[\textbullet] Open Name Manager with with \key{Ctrl} + \key{F3}
    \item[\textbullet] Name Box at top left corner (below "File")
\end{itemize}
\end{block}

However, the name is a reference to cells. In case we change the rows (columns) in our data we need to update the name.
}

\mframe{Matrix Calculation Functions}{
\begin{table}[]
\begin{tabular}{ll}
\rowcolor{Teal} 
{\color{White} Values in spreadsheet}& {\color{White} Values in matrix} \\
A cell &  A scalar\\
A column & A column vector \\
A row & A row vector \\
An array of $r$ rows and $c$ columns &  A $r \times c$ matrix
\end{tabular}
\end{table}

\begin{alertblock}{When dealing with matrices, Excel requires:}
\begin{itemize}
    \item[\textbullet] An array to store the results
    \item[\textbullet] \key{Ctrl} + \key{Shift} + \key{Enter}
\end{itemize}
\end{alertblock}
}

\mframe{Matrix Addition and Subtraction}{
Example from Lecture 1 handout p.50
$$A = \begin{pmatrix}
 0.3 & 0.6 \\
−0.1 & 0.7 \\
\end{pmatrix},
B = \begin{pmatrix}
0.2 & −0.1\\
0 & 0.3\\
\end{pmatrix}$$
$$A + B = \begin{pmatrix}
0.3 + 0.2 & 0.6 + 0.1 \\
−0.1 + 0 & 0.7 + 0.3\\
\end{pmatrix} = 
\begin{pmatrix}
0.5 &0.5 \\
−0.1& 1.0\\
\end{pmatrix}$$
$$A - B = \begin{pmatrix}
0.3 - 0.2 & 0.6 - 0.1 \\
−0.1 - 0 & 0.7 - 0.3\\
\end{pmatrix} = 
\begin{pmatrix}
0.1 & 0.7 \\
−0.1 & 0.4 \\
\end{pmatrix}$$
\begin{exampleblock}{Operation in Excel}
\key{+} and \key{-} do the same, but you need store the results in arrays and use combination of \key{Ctrl} + \key{Shift} + \key{Enter}
\end{exampleblock}
}

\mframe{Matrix Multiplication}{
Matrix multiplied by a scalar, e.g. p.51
$$A = \begin{pmatrix}
0.3 & 0.6 \\
−0.1 & 0.7 \\
\end{pmatrix}$$
$$2A = \begin{pmatrix}
2 \times 0.3 & 2 \times0.6 \\
2 \times−0.1 & 2 \times0.7 \\
\end{pmatrix} =
\begin{pmatrix}
0.6 & 1.2\\
−0.2 & 1.4 \\
\end{pmatrix}$$

\begin{exampleblock}{Operation in Excel}
\key{*} does the same, but you need store the results in arrays and use combination of \key{Ctrl} + \key{Shift} + \key{Enter}
\end{exampleblock}
}

\mframe{Matrix Multiplication}{
Matrix multiplied by a matrix, e.g. p.53
$$A_{2 \times 3} = \begin{pmatrix}
1 &2 &3 \\
2 &1 &2 \\
\end{pmatrix},
B_{3 \times 2} = \begin{pmatrix}
1 &2\\
3 &6\\
2 &2\\
\end{pmatrix}$$
\begin{aligned}
A_{2 \times 3} \times B_{3 \times 2} &= 
\begin{pmatrix}
1\times1 + 2\times3 + 3\times2 &1\times2 + 2\times6 + 3\times2 \\
2\times1 + 1\times3 + 2\times2 &2\times2 + 1\times6 + 2\times2 \\
\end{pmatrix}\\
&=
\begin{pmatrix}
13&	20\\
9&	14 \\
\end{pmatrix}
\end{aligned}

\begin{alertblock}{Remark}
The number of columns in $\mathbf{X}$ equals the number of rows $\mathbf{Y}$.
\end{alertblock}

\begin{exampleblock}{Operation in Excel}
 MMULT(X,Y) for $\mathbf{X_{m \times k}} \times \mathbf{Y_{k \times n}}$
\end{exampleblock}
}

\mframe{Matrix Inverse}{
 Matrix multiplication is not commutive in general: \vspace{-10pt}
$$AB \neq BA$$
 For this reason, we define inverse  \vspace{-10pt}
 $$A A^{-1} = I = A^{-1}A $$ to distinguish $ABA^{-1}$ (``division from left" ) and $A^{-1}AB = B$ (``division from left" )

 $A^{-1}$ is obtained from solving equation $ Ax = I$
 
 \begin{block}{Shortcut}
 For a $2 \times 2$ matrix $X=
     \begin{pmatrix}
     a & b\\
     c & d \\
     \end{pmatrix}$, X^{-1}= \dfrac{1}{ad-bc} \begin{pmatrix}
     d & −b \\
     −c & a \\
     \end{pmatrix}
 \end{block}

}

\mframe{Matrix Inverse (cont.)}{
\begin{alertblock}{Remark}
 $\mathbf{X}$ should be a square and full-rank matrix to be invertible.
\end{alertblock}
\begin{exampleblock}{Operation in Excel}
 MINVERSE(X) for $\mathbf{X^{-1}_{n \times n}}$
 \end{exampleblock}
}

\mframe{Matrix Transpose}{
 Matrix transpose, e.g. p.55
 $$A = \begin{pmatrix}
1 &2\\
7 &3\\
1 &6\\
\end{pmatrix}, \ A' = \begin{pmatrix}
1 &7 &1\\
2 &3 &6\\
\end{pmatrix}$$
$A$ is a $3 \times 2$ matrix and $A'$ is a $2 \times 3$ matrix

\begin{exampleblock}{Operation in Excel}
TRANSPOSE(X) for $\mathbf{X^{'}_{n \times k}}$
\end{exampleblock}
}


\mframe[OnlyMe]{Double Column}{
\begin{columns}
\column{0.5\textwidth}
This is a text in first column.
$$E=mc^2$$
\begin{itemize}
\item First item
\item Second item
\end{itemize}
 
\column{0.5\textwidth}
This text will be in the second column
and on a second tought this is a nice looking
layout in some cases.
\end{columns}
}


\mframe[OnlyMe]{Blocks}{

In this slide, some important text will be
\alert{highlighted} because it's important.
Please, don't abuse it.
 
\begin{block}{Remark}
Sample text
\end{block}
 
\begin{alertblock}{Important theorem}
Sample text in red box
\end{alertblock}
 
\begin{examples}
Sample text in green box. The title of the block is ``Examples".
\end{examples}
}


\end{document}